\documentclass{article}
\usepackage{polski}
\usepackage[utf8]{inputenc}

\title{Systemy wieloagentowe}
\author{Kajetan Zimniak i Bartosz Górka }
\date{Październik 2019}

\usepackage{natbib}
\usepackage{graphicx}

\begin{document}

\maketitle

\section{Część pierwsza}
\subsection{Opis modelowanego zjawiska / systemu}
W ramach projektu zamodelowany został ruch drogowy w mieście. Głównym celem symulacji jest zbadanie wielkości utrudnień drogowych (korki) i szybkości przejazdu w zależności od rozwiązań infrastruktury drogowej.

\subsection{Opis koncepcyjny modelu}
Powszechnie wiadomo, że główna przyczyna powstania korków to zbyt duże nagromadzenie pojazdów na danym odcinku trasy, wykraczające ponad możliwości przepustowe odcinka. W przypadku zmiany pasa jazdy, wymuszenia zmniejszenia prędkości przez podróżującego tym pasem, możemy zaobserwować zjawisko ``fali uderzeniowej''.

Ponadto istnieje szeroko promowane zjawisko rozwijania komunikacji miejskiej celem ograniczenia liczby pojazdów. Ma to wpływ zarówno na liczbę pojazdów w mieście, jak i powietrze (mniej pojazdów to potencjalnie mniej zanieczyszczeń). Aby było możliwe jej sprawne funkcjonowanie powstają specjalne ułatwienia drogowe jak tzw. bus-pasy, które jednocześnie mogą utrudniać ruch innym pojazdom.

Wiemy także o negatywnym wpływie sygnalizacji drogowej na płynność przemieszczania się. Konieczność wyhamowania i ponownego ruszenia zakłóca płynność jazdy.

\subsection{Założenia upraszczające}
W symulacji nie zakładamy obecności pieszych poruszających się poza miejscami do tego przeznaczonymi. Piesi mimo obecności w symulacji nie będą przekraczać ``fizycznie'' przejść dla pieszych, lecz wymuszać na kierowcach zatrzymanie się przed takimi miejscami. Samo ich pojawianie się będzie odwzorowane jako rozkład prawdopodobieństwa.

Dodatkowo przyjmujemy założenie o nie zatrzymywaniu się samochodów z przyczyn niezależnych od innych użytkowników ruchu (np. nie dopuszczamy sytuacji, w której pojazd zwolni, ponieważ pasażer chce przeczytać reklamę uliczną). Pojazdy i piesi poruszają się z maksymalnie dozwoloną prędkością.

\subsection{Lista typów użytych agentów, wraz z ich opisem}
\begin{itemize}
    \item Chodnik, przejście dla pieszych, jezdnia, rondo, skrzyżowanie drogowe - \textit{patch} - wyznaczają możliwy obszar przemieszczania się pojazdów oraz pieszych.
    \item Samochody - \textit{turtle} - mogą poruszać się tylko po zwykłych jezdniach.
    \item Pojazdy uprzywilejowane (np. autobus) - \textit{turtle} - mogą poruszać się zarówno po bus-pasach i zwykłych jezdniach.
    \item Piesi - \textit{turtle} - po pojawieniu się w okolicy samochodu powoduje jego obowiązkowe zatrzymanie, a tym samym zablokowanie ruchu. Mają pierwszeństwo przed pozostałymi pojazdami.
\end{itemize}

\subsection{Parametry modelu, wraz z ich opisem}
\begin{itemize}
    \item Liczba samochodów - oznacza sumaryczną liczbę pojazdów poruszających się po drodze, dla których bada się wielkość tworzących się korków.
    \item Średnia prędkość samochodów - to miernik wielkości korków. Pozwala porównać szybkość przejazdu przez miasto w zależności od symulowanej sytuacji.
    \item Średnie przyspieszenie - jak szybko pojazd może uzyskać swoją maksymalną prędkość 
    \item Prawdopodobieństwo pojawienia się pieszego w okolicy przejścia dla pieszych
    \item Ilość pasów ruchu - jak bardzo rozbudowana jest infrastruktura w mieście
    \item Czas trwania światła zielonego bądź czerwonego - definiuje jak wiele pojazdów może przejechać przez skrzyżowanie (zakładając konieczność stopniowego przyspieszenia od prędkości zerowej)
\end{itemize}

\subsection{Hipotezy badawcze}
\begin{itemize}
    \item W ramach projektu badany będzie wpływ na wielkość ruchu drogowego (korków) zamiany skrzyżowań na ronda. Przed symulacją uważa się, że ronda są korzystniejsze dla kierowców.

    \item Kolejną hipotezą jest zmniejszenie się korków poprzez budowanie naziemnych przejść dla pieszych, co izoluje ruch samochodowy od pieszego.

    \item Trzecim elementem jest badanie sensowności wyznaczania tzw. bus-pasów w kontekście korków dla ogółu użytkowników dróg w mieście.
\end{itemize}
\end{document}

------ Notatki po spotkaniu 29.10.2019 ----
Rondo jako rzeczywiste:
- krawdrat z ulic jednokierunkowych
- mozna jechac dookola w jednym kierunku

Można by zasugerować pasażerów
- średnio x pasażerów na jeden tik symulacji
- chcą oni przejechać skąd do jakiegoś celu
- albo samochodem albo autobusem
- zobaczyć jaka jest średnia prędkość przejazdu komunikacji miejskiej i samochodu - rozkład prawdopodobieństwa ich preferowania komunikacji.
    - nie mają samochodu, mają samochód
    - jak wpływa czas
    - jak będzie dużo pasażerów chętnych to pewnie dużo autobusów będzie
    - dynamiczna zmiana liczby pasażerów w zależności od czasu przejazdu (zobaczyć jaki będzie poziom).
        - pasażerowie jako lista żółwi w liście (w pamięci autobusu)
        - aktualizacja ich pozycji

